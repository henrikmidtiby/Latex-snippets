\documentclass[a4paper]{book}


\makeatletter
\renewcommand{\l@chapter}{\@dottedtocline{0}{0em}{1.4em}}
\renewcommand{\part}[1]{\addcontentsline{toc}{part}{#1}{\cleardoublepage\vspace{10pt}\par\noindent\Huge\bf#1}\vspace{10pt}\@afterheading\par}
\newcommand{\mychapter}[1]{\stepcounter{chapter}\addcontentsline{toc}{chapter}{Chapter \thechapter: #1}\vspace{20pt}{\noindent\Large\bf Chapter \thechapter: #1}\vspace{10pt}\@afterheading\par}




\newcommand\tempmypart{\@startsection {part}{0}{\z@}%
     {-2.5ex \@plus -1ex \@minus -.2ex}%
     {1.3ex \@plus.2ex}%
    {\cleardoublepage\bfseries\huge}}
\newcommand\tempmychapter{\@startsection {chapter}{1}{\z@}%
     {-2.5ex \@plus -1ex \@minus -.2ex}%
     {1.3ex \@plus.2ex}%
    {\Large\bfseries}}
\newcommand{\mypart}[1]{
	\refstepcounter{part}
 	\tempmypart*{#1}
 	\addcontentsline{toc}{part}{#1}}
\renewcommand{\mychapter}[1]{
 	\refstepcounter{chapter}
 	\tempmychapter*{Chapter \thechapter:~#1}
 	\addcontentsline{toc}{chapter}{Chapter \thechapter: #1}}
% \@startsection {NAME}{LEVEL}{INDENT}{BEFORESKIP}{AFTERSKIP}{STYLE} 
%            optional * [ALTHEADING]{HEADING}
%    Generic command to start a section.  
%    NAME       : e.g., 'subsection'
%    LEVEL      : a number, denoting depth of section -- e.g., chapter=1,
%                 section = 2, etc.  A section number will be printed if
%                 and only if LEVEL gt or eq the value of the secnumdepth
%                 counter.
%    INDENT     : Indentation of heading from left margin
%    BEFORESKIP : Absolute value = skip to leave above the heading.  
%                 If negative, then paragraph indent of text following 
%                 heading is suppressed.
%    AFTERSKIP  : if positive, then skip to leave below heading,
%                       else - skip to leave to right of run-in heading.
%    STYLE      : commands to set style
%  If '*' missing, then increments the counter.  If it is present, then
%  there should be no [ALTHEADING] argument.  A sectioning command
%  is normally defined to \@startsection + its first six arguments.



\makeatother

\begin{document}




\chapter*{Contents}
\makeatletter
\@starttoc{toc}
\makeatother

\mypart{Introduction}
\mychapter{Scope and strategy}

\mypart{Part 1: Layered double hydroxides}
Part 1 will make up a theoretical overview over the group of layered double hydroxides. Chapter \ref{LDH} contains a description of the family of LDHs in general, Chapter 2 and 3 gives an overview over the structural aspects of LDHs. Further more will chapter 4 look into the interlayer region of LDHs. Chapter 5 describes the synthesis of LDHs while chapter 6 describes how it is possible to use LDHs as anion exchangers. Part 1 will be round off with a summery given in chapter 7 which summarises the essens needed to analyse and draw conclusions of the results obtained in the experimental part.  





\mychapter{A family of layered materials} \label{LDH}
\mychapter{Structural aspects}
\mychapter{The interlayer region}
\mychapter{Synthesis}
\section{Coprecipitatetion}
\section{The urea method}
\mychapter{Anion exchange}
\section{Mechanism for anionexchange}
\mychapter{Summary of part 1}
This part has covered the teoretical part of the structure, synthesis and anion exchange of layered double hydroxides. The terms layered double hydroxides (LDHs) used to designate synthetic or natural lamellar hydroxides with two or more kinds of metallic cations in the main layers and hydrated interlayer domains containing anionic species. Layered double hydroxides are minerals with a significant permanent anion exchange capacity in contrast to the well-known clay minerals, which have cation exchange properties. \cite{Olfs2009}. 
\mypart{Part 2: Characterisation Techniques}
\mychapter{Powder X-ray diffraction}

\mychapter{Test}
\label{test}

\ref{testTwo}

\mychapter{TestTwo}
\label{testTwo}

\ref{test}


% \tableofcontents

\end{document}